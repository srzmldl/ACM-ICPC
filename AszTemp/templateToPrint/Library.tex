\documentclass[landscape]{report}
\usepackage[body={26cm,18cm}, top=1.0cm, left=2.5cm]{geometry}
\geometry{papersize={29.7cm, 21.0cm}}
\usepackage{multicol}
\usepackage[toc]{multitoc}
\usepackage{xeCJK}
\usepackage{fontspec}
\newfontfamily\monaco{Monaco}
\usepackage{listings}
\usepackage{color}
\usepackage{minitoc}
\setCJKmainfont{Microsoft YaHei}

\definecolor{mygreen}{rgb}{0,0.6,0}
\definecolor{mygray}{rgb}{0.5,0.5,0.5}
\definecolor{mymauve}{rgb}{0.58,0,0.82}

\lstset{ %
  backgroundcolor=\color{white},   % choose the background color; you must add \usepackage{color} or \usepackage{xcolor}
  basicstyle=\small\monaco,        % the size of the fonts that are used for the code
  breakatwhitespace=false,         % sets if automatic breaks should only happen at whitespace
  breaklines=true,                 % sets automatic line breaking
  captionpos=b,                    % sets the caption-position to bottom
  commentstyle=\color{mygreen},    % comment style
  escapeinside={\%*}{*)},          % if you want to add LaTeX within your code
  extendedchars=true,              % lets you use non-ASCII characters; for 8-bits encodings only, does not work with UTF-8
  %frame=single,                    % adds a frame around the code
  keepspaces=true,                 % keeps spaces in text, useful for keeping indentation of code (possibly needs columns=flexible)
  keywordstyle=\color{blue},       % keyword style
  language=C++,                 % the language of the code
  otherkeywords={*,...},            % if you want to add more keywords to the set
  numbers=left,                    % where to put the line-numbers; possible values are (none, left, right)
  numbersep=5pt,                   % how far the line-numbers are from the code
  numberstyle=\tiny\monaco, % the style that is used for the line-numbers
  rulecolor=\color{black},         % if not set, the frame-color may be changed on line-breaks within not-black text (e.g. comments (green here))
  showspaces=false,                % show spaces everywhere adding particular underscores; it overrides 'showstringspaces'
  showstringspaces=false,          % underline spaces within strings only
  showtabs=false,                  % show tabs within strings adding particular underscores
  stepnumber=1,                    % the step between two line-numbers. If it's 1, each line will be numbered
  stringstyle=\color{mymauve},     % string literal style
  tabsize=2,                       % sets default tabsize to 2 spaces
}

\newcommand{\includecode}[2][c]{\lstinputlisting[caption=#2, escapechar=, language=#1]{#2}}

\begin{document} 
\begin{flushleft}

\dominitoc[n]
\title{ACM/ICPC Template}
\author{AsZ VincentLDL}
\date{\today}
\maketitle
\setcounter{secnumdepth}{3}
\tableofcontents
\newpage
\begin{multicols}{2}
\chapter{ java}
\section{ 读写}
\section{ 高精}

\chapter{ dp 优化}
\section{ 决策单调性优化}
\begin{itemize}
\item 形式: f[i] = f[j] + w[j, i]形式决策单调。
\item 一般打表找规律看决策是否单调。
\item 四边形不等式:w[i,j]+w[i + 1, j + 1] <= w[i + 1, j] + w[i + 1, j],则满足决策单调性。
\item 有时候不满足决策单调性,但是去掉完全不合法状态之后却可以满足。
\end{itemize}
 \includecode[c++]{hnoi2008toys.cpp}   

%\subsection{cout}
\section{ 单调队列优化以及写仙人掌图}
\begin{itemize}
\item 题目背景:仙人掌图上最长链
\item 形式:f[i] = max(g[j]) + w[i],w[i]单调, 可见,如果j<k, g[j]<g[k], 则j可以直接不考虑,所以此时维护g单调减的队列即可。
\item 仙人掌图找环:首先形成bfs树,发现有环,记pt,ph,然后选pt沿着pre走到跟,一路打时间戳;再从ph沿着pre走,就可以找到lca。pt,ph到lca的路径,加上pt-->ph就是基环了。
\end{itemize}
\includecode[c++]{ioi2008Island.cpp}
\section{  斜率优化}
\begin{itemize}

\item f[i] = min(a[i] * x[j] + b[i] * y[j])
\item 更好的理解:设P=f[i],则 y = (-a/b)x + P/b. 求满足要求的最小截距。或者通过各种转化,最优决策就是从无穷远朝原点移动,第一个碰上的点为最优决策点。
\item 很好的性质:所有最优决策一定在当前所有点构成的凸包上。(例如,在最优决策点划一条相应斜率的线,其余点均在该线上方,)
\end{itemize} 
\subsection{ 斜率以及x维都单调}
\paragraph{ }
想像斜率越来越大的直线往y正方向移动,第i次移动首次碰上k。对于以后的
    决策,因为斜率更大,那么在k之前,第i次移动没有碰上的点必然再也用不上了,
    所以可以维护一个单调队列。 下面例题是:把一个序列切开,每个部分权值是和平方加常数,求权值和最小值
    
\includecode[c++]{hdu3507.cpp}

\subsection{ 随便什么情况:cdq分治优化}

\begin{itemize}
\item 排序的顺序,凸壳的方向写之前一定要画清楚。
\item 这里归并排一维的序可以节省一个log的复杂度
\item cdq分治的顺序至关重要,千万不能乱。
\item  f[i]表示第i天手上的券全换成现金最多多少, 其中x[j],y[j]分别表示用f[j]的钱换成A,B券分别能有多少。 
\item  f[i] = max(max(A[i] * x[j] + B[i] * x[j], f[j]))
\item 就是经典的斜率优化问题咯。不用平衡树的话可以离线用cdq分治。先按照-A[i]/B[i]排序(具体大小顺序画一画就知道了)。
      solve(l, r)时需要按照下标lab大小分为两部分。然后solve(l,mid),同时主义归并把递散维x排好序。l~mid至mid+1~r 转移.
      最后solve(mid + 1, r),接着归并排好x就行了。
\end{itemize}

\includecode[c++]{cash.cpp}
\chapter{ 图论}
\section{ tarjan}
\subsection{ 2-sat}
\subsection{ 割顶,点双联通分量}
\subsection{ 桥,边双联通分量}
\section{ 平面图}
farmland那道题
\section{网络流}
\subsection{ dinic}
\subsection{ 费用流}
\subsection{ 常见模型}
\section{弦图}
\section{最小树形图}


\chapter{数据结构}
\section{splay}
\section{lct}
\section{可持久化线段树}
\section{ 点分治}
\section{ 树链剖分}

\chapter{字符串}
\section{ 后缀数组}
\section{后缀自动机}

\chapter{ 其他算法}
\section{ cdq 分治与读入优化}
\begin{itemize}

\item 不要排结构体,因为排结构体到时候还要排回来。
\item 线段树打时间戳不要memsize();
\item 在严格小的限制下,第二维排序的时候一定要双关键字排序
\item 这题是三维空间中,三个坐标都不减的最长链

\end{itemize}
\includecode[c++]{hdu4742.cpp}
\end{multicols}
\end{flushleft}
\end{document} 
